\documentclass[]{article}

\usepackage{hyperref}

% Title Page
\title{Towards Data Monetisation in Vaultage Environment}
\author{Alfa Yohannis}

\begin{document}
\maketitle

\begin{abstract}
  
\end{abstract}

\section{Background}
Most online applications available nowadays are centralised. Such architecture requires users to trust their application providers and grant them the right to process their data. User data is stored on the service providers' side causing users to have limited control over it. User data can be monetised and used for other purposes by malicious application providers without being known by users.

While users' data has been protected by law (e.g., GDPR \cite{GDPR}), such protection is limited as it requires law enforcement. Data misuses and breaches could still happen and are not prevented by the design of the technology. A breach in a centralised system could impact millions of users 
\cite{cambridgeanalytica,adobebreach,linkedinbreach,yahoobreach}.

In contrast to the approach of centralised systems, a decentralised system stores users' data, in users' personally-managed devices which gives users greater control over their data. A specific container is dedicated to help users to add, remove, and modify their data, and also to share data to other containers at different access levels, e.g, read-only, add-only, etc. We have seen such container in \textit{data pods} in the Solid framework \cite{solid} and or \textit{data vaults} as in \cite{mun2010datavault}. We use the terms \textit{data vault} or just \textit{vault} to refer to such container.  

We have developed Vaultage \cite{yohannis2019towards}, a model-based framework for the development of decentralised applications around user-managed data vaults.
It allows modelling both the data to be stored in a data vault and the set of data services that external clients can request from a vault.
From a model containing this information, Vaultage generates (1) a set of Java classes for the application's internal usage of the available data, and (2) a secure communication infrastructure that can be used to facilitate requests and responses between clients and data vaults of a specific application. 

A vault generated by Vaultage can also send queries in Epsilon Object Language \cite{epsilon} to other vaults to retrieve specific data which the can be used for different purposes, e.g., data analytics. This functionality is similar to the queries in federated databases \cite{db2} or Solid Pods/SparQL \cite{comunica} that allows retrieving data from different, autonomous data sources. One advantage of Vaultage is it supports propagation of queries which allows a vault to query a another vault even though they are not directly connected; the query is passed on by intermediary vaults. For example, a vault can send a query to a remote vault and then passed it on to the remote vault's sensors (other vaults in embedded systems) to retrieve the air temperature captured in real-time. 

With those functionalities, Vaultage are potential to bring new novelties into the interaction space between users and data while keeping privacy and security in mind. One of them  is the prospect to monetise data. For instance, the air temperature data captured in the previous example can be monetised as a charged service. Another example, a small grocery can sell their daily transaction data to different product distributors.  

Meanwhile, Cryptocurrency (e.g., Ethereum,\footnote{\url{https://ethereum.org/en/}} XRP\footnote{\url{https://xrpl.org/}}) and the technology around it (e.g., Interledger\footnote{\url{https://interledger.org/}})  grows tremendously in the last decade bringing significant changes in the way people interact with money. Payment becomes more decentralised, secure, and trustable than ever before, which in line to the vision of Vaultage.    

Some projects have integrated crypto-related technologies to add payment feature into decentralised vaults. For example, Becker \cite{becker2021monetising} and Meindertsma \cite{meindertsma2019web}  have extended Solid Pods to support user data monetisation using Ethereum and Interledger respectively. 

Integrating payment-related technology into Vaultage will enable it to automatically generate vaults that have a payment capability. This will facilitate developers to add the feature into their vault-based applications, which in turns, allowing end-users to monetise their data.   

\section{Aims and Objectives}
The purpose of this research is to add payment feature to 


\section{Research Method}


\section{Findings Distribution}
The source code of the artefact will be made open source and hosted on Github for public access. The findings will be published in a scientific journal or in a proceeding of a software engineering/information systems-related scientific conference.  

\section{Social Impact}

\section{Ethical Implications}

\section{Third-party Technical Features}
Only open-source libaries will be used in the artefact.  

\section{Collaboration with Other Platforms}
No collaboration with other platforms.

\section{Literature Review}

\subsection{Vaultage}
Vaultage is a tool and framework to generate decentralised peer-to-peer data vault applications. 

\subsection{Interledger}

\subsection{Web Monetisation}


\bibliography{references} 
\bibliographystyle{ieeetr}

%Including a thesis statement and well-scoped and testable research question
%
%Communicating a clear research methodology and suitable team Demonstrated 
%knowledge of past and/or existing web business models, including the technology that drives them
%
%Knowledge of existing research. A lit review would be a valuable addition to include in the supporting documents section of the application.
%
%A clear plan to share and distribute findings to support public learning including fair use licensing.
%
%How people who have been historically marginalized in technology and structurally excluded from financial opportunities will be included in your project.
%
%An understanding of any ethical implications of the work and plan to address.
%
%If your project’s technical requirements depends on third party technical features you should demonstrate a practical understanding that they will meet your needs.
%
%Collaborations with platforms should have letters of support.

\end{document}          
