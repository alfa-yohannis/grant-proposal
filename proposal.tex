\documentclass[]{report}

\usepackage{hyperref}

% Title Page
\title{Vaultage's Data Monetisation with Interledger}
\author{Alfa Yohannis}

\begin{document}
\maketitle

\begin{abstract}
  
\end{abstract}


\section{Introduction}
Most online applications available nowadays are centralised. Such architecture requires users to trust their application providers and grant them the right to process their data. User data is stored on the service providers' side causing users to have limited control over it. User data can be monetised and used for other purposes by malicious application providers without being known by users.

While users' data has been protected by law (e.g., GDPR \cite{GDPR}), such protection is limited as it requires law enforcement. Data misuses and breaches could still happen and are not prevented by the design of the technology. A breach in a centralised system could impact millions of users 
\cite{cambridgeanalytica,adobebreach,linkedinbreach,yahoobreach}.

In contrast to the approach of centralised systems, a decentralised system stores users' data, in users' personally-managed devices which gives users greater control over their data. A specific container is dedicated to help users to add, remove, and modify their data, and to share their data trusted users with different access levels, e.g, read-only, append-only, etc. We have seen such container in \textit{data pods} in the Solid framework \cite{solid} and or \textit{data vaults} as in \cite{mun2010datavault}.

\section{Thesis and Research Questions}

\section{Research Methodology}

\section{Literature Review}

\subsection{Vaultage}
Vaultage is a tool and framework to generate decentralised peer-to-peer data vault applications. 

\subsection{Interledger}

\subsection{Web Monetisation}

\section{Findings Distribution}
The source code of the artefact will be made open source and hosted on Github for public access. The findings will be published in a scientific journal or at a scientific software engineering/information systems conference.  

\section{Social Impact}

\section{Ethical Implications}

\section{Third-party Technical Features}
Only open-source libaries will be used in the artefact.  

\section{Collaboration with Other Platforms}
No collaboration with other platforms.

\bibliography{references} 
\bibliographystyle{ieeetr}

%Including a thesis statement and well-scoped and testable research question
%
%Communicating a clear research methodology and suitable team Demonstrated 
%knowledge of past and/or existing web business models, including the technology that drives them
%
%Knowledge of existing research. A lit review would be a valuable addition to include in the supporting documents section of the application.
%
%A clear plan to share and distribute findings to support public learning including fair use licensing.
%
%How people who have been historically marginalized in technology and structurally excluded from financial opportunities will be included in your project.
%
%An understanding of any ethical implications of the work and plan to address.
%
%If your project’s technical requirements depends on third party technical features you should demonstrate a practical understanding that they will meet your needs.
%
%Collaborations with platforms should have letters of support.

\end{document}          
