\documentclass[]{article}

\usepackage{hyperref}

% Title Page
\title{Research Proposal for `Grant for the Web'\footnote{\url{https://www.grantfortheweb.org/}} :\\Towards Data Monetisation\\in Vaultage Environment}
\author{
	Alfa Yohannis, Alfonso de la Vega, Dimitris Kolovos\\
	\\
	Department of Computer Science\\
	University of York
}
\date{}
\begin{document}
	\maketitle
	
	\begin{abstract}
	Vaultage is a model-based framework that aims to help developers to model, generate, and develop decentralised peer-to-peer data vault applications (similar to Inrupt's Solid Pods). Currently, Vaultage does not support data monetisation. This project aims to add a monetisation feature into Vaultage so that developers can create data vault applications that allow end-user to monetise their data. Ethereum, We plan to extend Vaultage to support direct or smart contract-managed payments via the Ethereum and XRP blockchains and Interledger network.
	\end{abstract}
	
	\section{Background}
	\label{sec:background}
	
	Most online applications available nowadays are centralised. Such architecture requires users to trust their application providers and grant them the right to process their data. User data is stored on the service providers' side causing users to have limited control over it. User data can be monetised and used for other purposes by malicious application providers without being known by users.
	
	While users' data has been protected by law (e.g., GDPR \cite{GDPR}), such protection is limited as it requires law enforcement. Data misuses and breaches could still happen and are not prevented by the design of the technology. A breach in a centralised system could impact millions of users 
	\cite{cambridgeanalytica,adobebreach,linkedinbreach,yahoobreach}.
	
	In contrast to the approach of centralised systems, a decentralised system stores users' data, in users' personally managed devices which gives users greater control over their data. A specific container is dedicated to helping users to add, remove, and modify their data, and also to share data to other containers at different access levels, e.g, read-only, add-only, etc. We have seen such container in \textit{data pods} in the Solid framework \cite{solid} and or \textit{data vaults} as in \cite{mun2010datavault}. We use the terms \textit{data vault} or just \textit{vault} to refer to such container.  
	
	We have developed Vaultage \cite{yohannis2019towards,vaultage}, a model-based framework for the development of decentralised applications around user-managed data vaults.
	It allows modelling both the data to be stored in a data vault and the set of data services that external clients can request from a vault.
	From a model containing this information, Vaultage generates (1) a set of Java classes for the application's internal usage of the available data, and (2) a secure communication infrastructure that can be used to facilitate requests and responses between clients and data vaults of a specific application. 
	
	A vault generated by Vaultage can also send queries \cite{vaultage-query} in Epsilon Object Language \cite{epsilon} to other vaults to retrieve specific data which can be used for different purposes, e.g., data analytics. This functionality is similar to the queries in federated databases \cite{db2} or Solid Pods/SparQL \cite{comunica} that allows retrieving data from different, autonomous data sources. One advantage of Vaultage is it supports the propagation of queries which allows a vault to query another vault even though they are not directly connected; the query is passed on by intermediary vaults. For example, a vault can send a query to a remote vault, and the latter pass it on to the remote vault's sensors (other vaults in embedded systems) to retrieve the air temperature captured in real-time. 
	
	With those functionalities, Vaultage has the capacity to bring new novelties into the interaction space between users and data while keeping privacy and security in mind. One of them is the prospect to monetise data. For instance, the air temperature data captured in the previous example can be monetised as a charged service. Another example, a small grocery can sell their daily transaction data to different product distributors.  
	
	Meanwhile, Cryptocurrency (e.g., Ethereum \cite{ethereum},  XRP \cite{xrpl}) and the technology around it (e.g., Interledger \cite{interledger})  grows tremendously in the last decade bringing significant changes in the way people interact with money. Payment becomes more decentralised, secure, and trustable than ever before, which is in line with the vision of Vaultage.    
	
	Some projects have integrated crypto-related technologies to add a payment feature into decentralised vaults. For example, Becker \cite{becker2021monetising} and Meindertsma \cite{meindertsma2019web}  have extended Solid Pods to support user data monetisation using Ethereum and Interledger respectively. 
	
	Integrating payment-related technology into Vaultage will enable it to automatically generate vaults that have a payment capability. This will facilitate developers to add the feature into their vault-based applications, which in turn, allowing end-users to monetise their data.   
	
	\section{Aim and Objectives}
	\label{sec:aim_and_objectives}
	The goal of this research is to extend Vaultage to be able to generate vaults that have a configurable payment feature in order to facilitate data monetisation. Due to the time constraint, we limit the monetisation methods that Vaultage will support only to \emph{one-time buy} (single payment per unit of data) and \emph{subscription} (time-limited consumption) methods. In order to achieve the goal, we have set the following objectives:
	\begin{enumerate}
		\item \textbf{Sending Payments}. Build a prototype that extends Vaultage to automatically generate infrastructure code that enables vaults to interact with wallets and send payments in Ethereum, XRP, and Interledger networks. Note that the purpose of Interledger used in this research is to support cross-currency payments, such as from Ethereum to XRP or vice versa.
		\item \textbf{Smart Contract Creation}. Automatic creation of smart contracts in Ethereum or escrows\footnote{Escrows are viewed as smart contracts further in this text.} \cite{xrpl-escrow} in XRP as the representation of agreement and payment mechanisms between a buyer and seller. 
		\item \textbf{Evaluation}. Evaluate the functionality and performance of vaults generated by Vaultage---the prototype. 
		\item \textbf{Publication}. Write a paper and documentation to publish findings and the prototype.
	\end{enumerate}  
	
	\section{Methodology}
	\label{Methodology}
	In this section, we present in detail the methods that we plan to execute in order to satisfy the objectives in Section \ref{sec:aim_and_objectives}. 
	
	\textbf{Sending Payments}. 
	We will integrate Vaultage with Ethereum, XRP, and Interledger in order to allow accessing wallets and sending payments between vaults. Since Vaultage is dominantly built around Java technology, using existing Java-based APIs that can make use of these payment technologies will speed up the development of the prototype. We have identified \cite{web3j}, Xrpl4j \cite{xrpl4j}, and Hyperledger Quilt \cite{quilt} as the APIs that we will use for interfacing with Ethereum, XRP, and Interledger respectively. Moreover, the interface of the prototype will be web-based to allow web monetisation as required by \emph{the Grant for the Web}'s call for proposals \cite{grant-for-the-web}. We allocate one and a half months for this objective. 
	
	\textbf{Smart Contract Creation}. 
	In order to create trust between buyers and sellers and automate transactions, we also plan to make use of smart contracts as the representation of agreement and payment mechanisms used by the business model in context. We will develop templates of Ethereum smart contracts and XRP escrows to represent the one-time buy and subscription payment mechanisms. The templates should be configurable according to the parameters defined when creating the smart contracts.    
	They will be automatically deployed in the respective networks using the Web3j and XRP escrow APIs once a buyer and seller give their confirmation.
	
	The challenge comes when we want to integrate the Enthereum smart contracts and XRP escrows with Interledger. For now, Interledger only supports account-to-account transfer; it has not supported account-to-smart contract payment. 
	Thus, Interledger should be extended. 
	This can be done through developing a new plugin \cite{interledger-plugin} for Interledger or modifying Interledger \cite{interledger-rs} and its XRP and Ethereum settlement engines \cite{xrp-settlement,eth-settlement} to support such function. 
	We estimate two months should be allocated for this objective.
	For future work, Interledger could also be extended to support smart contract-to-account and  smart contract-to-smart contract money transfers. 
	
	\textbf{Evaluation}. 
	We will also evaluate the correctness of the functionalities in the previous two objectives. Vaultage should be able to provide, or generate if needed, infrastructure code for vaults to access wallets (e.g., get addresses, read balances, etc.). The generated vaults can send money to the right recipients and receive payments in the correct amounts via Ethereum, XRP, and Interledger networks. The generated vaults should be able to create the right smart contracts that are able to function properly in automating payment mechanisms---one-time buy and subscription.
	
	We will also measure the performance of the generated vaults on how much time on average they need to complete single payments and perform a whole scenario of using the smart contracts---from creation to termination. 
	
	For testing, we will use Ethereum's Ganache \cite{ganache} and Ropsten \cite{ropsten} test networks, XRP test network \cite{xrp-testnet}, and the minimal local ILP network \cite{ilp-testnet} to test cross-currency payments. We allocate one and a half months for the evaluation. 
	
	\textbf{Publication}. The plan for publication is presented in the next section---Section \ref{sec:findings_distribution}. One month is allocated for this objective.   
	
	\section{Findings Distribution}
	\label{sec:findings_distribution}
	The source code of the prototype will be made open-source and hosted on Github for public access. The findings will be submitted to a scientific journal or in a proceeding of a software engineering/information systems-related scientific conference (a preprint version will also be made available due to uncertainty of review process time).   
	
	\section{Social Impact}
	Developers can use Vaultage to develop vault-based applications. The applications will have a payment feature that can facilitate end-users to monetise their data. For example, small groceries can monetise their daily transactions to distributors, social network users can sell their online posts, and embedded system owners can lease their machines.     
	
	\section{Ethical Implications}
	No direct ethical implication is identified in this research.
	
	\section{Third-party Technical Features}
	Only open-source libraries will be used in the prototype.  
	
	\section{Collaboration with Other Platforms}
	No collaboration with other platforms.
	
	\section{Team}
	The team for this project consists of three people. One principal investigator (PI) and two research staff. The PI is responsible for the preparation, conduct, administration, and cooperative agreement of the research project and grant. The research staff is responsible for developing the prototype and its publication. 
	
	%\section{Literature Review}
	%
	%\subsection{Vaultage}
	%Vaultage is a tool and framework to generate decentralised peer-to-peer data vault applications. 
	%
	%\subsection{Interledger}
	%
	%\subsection{Web Monetisation}
	
	
	\bibliography{references} 
	\bibliographystyle{ieeetr}
	
	%Including a thesis statement and well-scoped and testable research question
	%
	%Communicating a clear research methodology and suitable team Demonstrated 
	%knowledge of past and/or existing web business models, including the technology that drives them
	%
	%Knowledge of existing research. A lit review would be a valuable addition to include in the supporting documents section of the application.
	%
	%A clear plan to share and distribute findings to support public learning including fair use licensing.
	%
	%How people who have been historically marginalized in technology and structurally excluded from financial opportunities will be included in your project.
	%
	%An understanding of any ethical implications of the work and plan to address.
	%
	%If your project’s technical requirements depends on third party technical features you should demonstrate a practical understanding that they will meet your needs.
	%
	%Collaborations with platforms should have letters of support.
	
\end{document}          
